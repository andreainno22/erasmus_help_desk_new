\chapter{Database}

\section{Schema logico}
Tabelle principali:
\begin{itemize}
  \item \texttt{universities}: anagrafica atenei (\texttt{id}, \texttt{university\_name}, email, ...)
  \item \texttt{uploaded\_documents}: documenti caricati (\texttt{id}, \texttt{university\_id}, \texttt{document\_type}, \texttt{file\_path}, \texttt{stored\_filename}, ...)
\end{itemize}

\section{Tipi di documento}
\begin{itemize}
  \item \texttt{erasmus\_call}
  \item \texttt{destinazioni}
  \item \texttt{corsi\_erasmus}
\end{itemize}

\section{Esempio di query}
Ricerca corsi per università di destinazione:
\begin{lstlisting}[style=code]
SELECT d.*
FROM uploaded_documents d
JOIN universities u ON d.university_id = u.id
WHERE d.document_type = 'corsi_erasmus'
  AND d.is_active = 1
  AND (
    LOWER(u.university_name) = LOWER(?)
    OR LOWER(u.university_name) LIKE LOWER(?)
    OR LOWER(?) LIKE '%' || LOWER(u.university_name) || '%'
  )
ORDER BY d.upload_date DESC
LIMIT 1;
\end{lstlisting}
